\documentclass{beamer}
% Copyright 2015 by Do Phan Thuan

% Loại mẫu slice
%\usetheme{AnnArbor}
%\usetheme{Antibes}
\usetheme{Boadilla}
%\usetheme{CambridgeUS}
%\usetheme{Hannover}

% Ký tự tiếng Việt
\usepackage[utf8]{vietnam}
\usepackage[utf8]{inputenc}
% Công thức toán
\usepackage{amsmath,amsthm,amssymb,epsfig}
% Chèn ảnh
\usepackage{graphicx}
% Chèn đường dẫn 
\usepackage{url}

% Vẽ đồ thị
\usepackage{pgfplots}

% Insert code
\usepackage{listings}
\lstset{language=C++,
   %keywords={break,case,catch,continue,else,elseif,end,for,function,
   %   global,if,otherwise,persistent,return,switch,try,while},
   basicstyle=\ttfamily,
   keywordstyle=\color{blue},
   commentstyle=\color{red},
   stringstyle=\color{dkgreen},
   frame=lrtb,
   %frame=5 pt,
   numbers=left,
   numberstyle=\tiny\color{gray},
   stepnumber=1,
   numbersep=10pt,
   backgroundcolor=\color{white},
   tabsize=4,
   showspaces=false,
   showstringspaces=false}
% Tô mầu cho bảng
\usepackage{colortbl}


\usepackage{color}

\definecolor{dkgreen}{rgb}{0,0.6,0}
\definecolor{gray}{rgb}{0.5,0.5,0.5}
\definecolor{mauve}{rgb}{0.58,0,0.82}
  
\definecolor{Xanh}{rgb}{0,0.5,1}
\definecolor{Do}{rgb}{1,0.25,0}
\definecolor{Vang}{rgb}{1,1,0}
\definecolor{Datroi}{rgb}{0,0,1}
% Vẽ hình
\usepackage{tikz}
\usetikzlibrary{arrows,shapes}
% Vẽ mạch điện
\usepackage[siunitx,european resistors]{circuitikz}

% multirow
\usepackage{multirow}

\usepackage{pbox}

% Tô mầu cho bảng
\usepackage{colortbl}
\definecolor{Xanh}{rgb}{0,0.5,1}
\definecolor{Do}{rgb}{1,0.25,0}
\definecolor{Vang}{rgb}{1,1,0}
\definecolor{Datroi}{rgb}{0,0,1}

% Một vài ký hiệu thường dùng
\def\R{{\mathbb R}}
\def\N{{\mathbb N}}
\def\X{{\mathcal X}}
\def\Y{{\mathcal Y}}
\def\F{{\mathcal F}}
\def\P{{\mathcal P}}
\def\E{{\mathbb E}}
\def\I{{\mathbb I}}
\def\sign{{\rm sign}}

% Xác định khoảng dãn trong bảng
\renewcommand\arraystretch{1.2}

% a few macros
\newcommand{\bi}{\begin{itemize}}
\newcommand{\ei}{\end{itemize}}
\newcommand{\ig}{\includegraphics}
\newcommand{\subt}[1]{{\footnotesize \color{subtitle} {#1}}}

% named colors
\definecolor{offwhite}{RGB}{249,242,215}
\definecolor{foreground}{RGB}{255,255,255}
\definecolor{background}{RGB}{24,24,24}
\definecolor{title}{RGB}{107,174,214}
\definecolor{gray}{RGB}{155,155,155}
\definecolor{subtitle}{RGB}{102,255,204}
\definecolor{hilight}{RGB}{22,155,104}
\definecolor{vhilight}{RGB}{255,111,207}
\definecolor{lolight}{RGB}{155,155,155}
%\definecolor{green}{RGB}{125,250,125}

% Minted
%\usepackage{minted}
%\usemintedstyle{monokai}
%\newminted{cpp}{fontsize=\footnotesize}

% Graph styles
\tikzstyle{vertex}=[circle,fill=black!50,minimum size=15pt,inner sep=0pt, font=\small]
\tikzstyle{selected vertex} = [vertex, fill=red!24]
\tikzstyle{edge} = [draw,thick,-]
\tikzstyle{dedge} = [draw,thick,->]
\tikzstyle{weight} = [font=\scriptsize,pos=0.5]
\tikzstyle{selected edge} = [draw,line width=2pt,-,red!50]
\tikzstyle{ignored edge} = [draw,line width=5pt,-,black!20]

%gets rid of bottom navigation bars
\setbeamertemplate{footline}[frame number]{}

%gets rid of bottom navigation symbols
%\setbeamertemplate{navigation symbols}{}

%gets rid of footer
%will override 'frame number' instruction above
%comment out to revert to previous/default definitions
%\setbeamertemplate{footline}{}

% Tác giả, Tiêu đề, vân vân
\title[]{{\huge \bf Chương 3.\\ Cấu trúc dữ liệu nâng cao} \\
\large Lập trình thi đấu}
\author[Đỗ Phan Thuận]{
Đỗ Phan Thuận% \inst{1} 
}

\institute[]{
%\inst{1}% 
Bộ môn Khoa Học Máy Tính, Viện CNTT \& TT, \\
Trường Đại Học Bách Khoa Hà Nội.
}

\logo{\includegraphics[scale=0.05]{hust.jpg} \vspace{220pt}}

\begin{document}

\begin{frame}
\titlepage
\end{frame}

\begin{frame}{Nội dung}
\tableofcontents
\end{frame}

\section{Lý thuyết-15'}
\subsection{Tổng quan-2'}
\subsection{Các kỹ thuật-10'}
\subsection{Các ứng dụng-3'}
\section{Bài tập thực hành-10'/bài}
\subsection{Giới thiệu và Mô tả bài toán-1'}
\subsection{Thuật toán giải-4'}
\subsection{Chứng minh tính đúng đắn và Độ phức tạp-2'}
\subsection{Mô tả cài đặt-1'}
\subsection{Mô tả bộ dữ liệu thử nghiệm-1'}
\subsection{Demo chương trình/kết quả: chụp ảnh kết quả chạy chương trình chấm Themis-1'}
\section*{Tài liệu tham khảo}

% TODO: Book
\begin{frame}{Sách tham khảo}
    \vspace{20pt}

    \bi
        \item {\color{hilight}Competitive Programming} by Steven Halim
        \item[] (Sử dụng bản 2 hoặc bản 3)
        \vspace{10pt}
        \item {\color{hilight}Bài giảng Chuyên đề} by Lê Minh Hoàng
    \ei
\end{frame}

% TODO: Overview
\begin{frame}{Các bài giảng}
    \bi
        \item Dự kiến các chủ đề
    \ei

    {
        \tiny

        \begin{center}
            \begin{tabular}{cl|ll}
                Thứ tự. & Thời gian & Chủ đề & Hoạt động \\
                \hline
                1 & 2005 & Giới thiệu & \\
                2 & 2005 & Cấu trúc dữ liệu và thư viện & \\
                3 & 2005 & Cấu trúc dữ liệu nâng cao & \\
                4 & 2005 & Sơ đồ giải bài toán & \\
                5 & 2005 &\pbox{20cm}{\vspace{2pt}Tham lam \\Quy hoạch động} & Bài tập thực hành I \\
                  & 2005 & & \\
                6 & 2005 & Quy hoạch động & \\
                7 & 2005 & Đồ thị vô hướng & \\
                8 & 2005 & Đồ thị & \\
                9 & 2005 & Luồng trong mạng & \\
                10 & 2005 & Toán học & Bài tập thực hành II \\
                   & 2005 & & \\
                11 & 2005 & Xâu & \\
                12 & 2005 & Hình học & \\
                13 & 2005 & & Kiểm tra \\
            \end{tabular}
        \end{center}
    }
    
\end{frame}

% TODO: What kind of problems we're dealing with: description/input/output
\begin{frame}{Mẫu đề bài}
    \bi
        \item Mẫu chuẩn trong hầu hết các kỳ thi bao gồm
        
            \bi
                \item Mô tả bài toán
                \item Mô tả định dạng dữ liệu vào
                \item Mô tả định dạng kết quả ra
                \item Ví dụ Dữ liệu vào/Kết quả ra
                \item Giới hạn thời gian theo giây
                \item Giới hạn bộ nhớ theo bytes/megabytes
            \ei
        \item Yêu cầu viết chương trình giải bài toán đúng càng nhiều bộ dữ liệu càng tốt. Mặc định là dữ liệu vào không cần kiểm tra tính đúng đắn
        \item Chương trình không được chạy quá giới hạn thời gian và giới hạn bộ nhớ
    \ei
\end{frame}

% TODO: Example problems/solutions
\begin{frame}{Bài toán ví dụ}
    \vspace{10pt}
    {\footnotesize\color{title}Mô tả bài toán}\\
    Viết chương trình nhân hai số nguyên.

    \vspace{10pt}

    {\footnotesize\color{title}Mô tả dữ liệu vào}\\
    Dòng đầu tiên chứa một số nguyên $T$, với $1\leq T \leq
    100$, là số lượng bộ test. $T$ dòng tiếp theo, mỗi dòng chứa một test. Mỗi test bao gồm 2 số nguyên $A,B$,
    với $-2^{20} \leq A,B \leq 2^{20}$, cách nhau ít nhất một dấu cách.

    \vspace{10pt}

    {\footnotesize\color{title}Mô tả kết quả ra}\\
    Kết quả ghi ra mỗi dòng tương ứng với một test chứa một số là giá trị $A\times B$.
\end{frame}

\begin{frame}{Bài toán ví dụ}
    \vspace{10pt}

    \begin{center}
        \begin{tabular}{|l|l|}
            \hline
            {\footnotesize Ví dụ dữ liệu vào} & {\footnotesize Dữ liệu kết quả ra} \\
            \hline
            \begin{minipage}{80pt}
\vspace{10pt}
\ttfamily
4\\
3 4\\
13 0\\
1 8\\
100 100\\
            \end{minipage}
&
\begin{minipage}{80pt}
\vspace{10pt}
\ttfamily
12\\
0\\
8\\
10000\\
\end{minipage}
\\
            \hline
        \end{tabular}
    \end{center}

\end{frame}

\begin{frame}[fragile]{Lời giải ví dụ}
\begin{lstlisting}{frame=single}
#include <iostream>
using namespace std;
int main() {
    int T;
    cin >> T;
    for (int t = 0; t < T; t++) {
        int A, B;
        cin >> A >> B;
        cout << A * B << endl;
    }
    return 0;
}
\end{lstlisting}

    \bi
        \onslide<2->{\item Lời giải này có đúng không? \onslide<5->{{\color{vhilight}KHÔNG!}}}
        \onslide<3->{\item Điều gì xảy ra nếu $A = B = 2^{20}$? \onslide<4->{Kết quả ra $0$...}}
    \ei
\end{frame}



\end{document}